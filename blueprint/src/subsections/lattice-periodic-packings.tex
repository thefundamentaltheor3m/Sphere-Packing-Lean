\subsection{Lattices and Periodic packings}

\subsubsection{On Lattices}

We will begin by stating some basic facts about lattices.

\begin{definition}\label{IsZlattice}\lean{IsZlattice}\leanok
  We say that an additive subgroup $\Lambda \leq \R^d$ is a \emph{lattice} if it is discrete and its $\R$-span contains all the elements of $\R^d$.
\end{definition}

There are several useful properties of lattices that already exist in Mathlib. We will state them here without proof.

\begin{lemma}\label{Zlattice.module_free}\lean{Zlattice.module_free}\uses{IsZlattice}\leanok
  Any lattice is a free $\Z$-module.
\end{lemma}
\begin{proof}\leanok
  See \verb|theorem Zlattice.module_free| in \verb|Mathlib.Algebra.Module.Zlattice.Basic|.
\end{proof}

\begin{lemma}\label{Zlattice.rank}\lean{Zlattice.rank}\uses{IsZlattice}\leanok
  Any lattice in $\R^d$ has $\Z$-rank $d$.
\end{lemma}
\begin{proof}\uses{Zlattice.module_free}
  See \verb|theorem Zlattice.rank| in \verb|Mathlib.Algebra.Module.Zlattice.Basic|.
\end{proof}

\begin{lemma}\label{Basis.ofZlatticeBasis}\lean{Basis.ofZlatticeBasis}\uses{IsZlattice}\leanok
  Any $\Z$-basis of $\Lambda$ is an $\R$-basis of $\R^d$.
\end{lemma}
\begin{proof}\uses{Zlattice.module_free, Zlattice.rank}\leanok
  See \verb|def Basis.ofZlatticeBasis| in \verb|Mathlib.Algebra.Module.Zlattice.Basic|.
\end{proof}

\subsubsection{Lattice-Periodic Packings and Lattice Packings}

We will now define periodic packings and lattice packings, which will be of great interest to us as we move forward.

\begin{definition}\label{PeriodicSpherePacking}\lean{PeriodicSpherePacking}\uses{SpherePacking, IsZlattice}\leanok
  We say that a sphere packing $\Pa(X)$ is ($\Lambda$-)\emph{periodic} if there exists a lattice $\Lambda \subset \R^d$ such that for all $x \in X$ and $y \in \Lambda$, $x + y \in X$ (ie, $X$ is $\Lambda$-periodic).
\end{definition}

\begin{lemma}\label{PeriodicSpherePacking.toSpherePacking}\lean{PeriodicSpherePacking.toSpherePacking}\uses{PeriodicSpherePacking, SpherePacking}\leanok
  Every periodic sphere packing is a sphere packing.
\end{lemma}
\begin{proof}
  Mathematically, this lemma is hardly worth mentioning. We only do so to underscore the automatically constructed forgetful map \verb|PeriodicSpherePacking.toSpherePacking| in Lean.
\end{proof}

\begin{lemma}\label{PeriodicSpherePacking.instAddAction}\lean{PeriodicSpherePacking.instAddAction}\uses{PeriodicSpherePacking}\leanok
  If $\Pa(X)$ is a $\Lambda$-periodic sphere packing, then $\Lambda$ acts on $X$ by translation.
\end{lemma}
\begin{proof}\leanok
  This is immediate from the definition of a periodic sphere packing.
\end{proof}

\begin{definition}
  If $\Lambda$ is a lattice, we call the $\Lambda$-periodic packing $\Pa(\Lambda)$ with centres at points in $\Lambda$ a lattice packing.
\end{definition}

\begin{lemma}\label{SpherePacking.density of periodic packing}\notready
  If $X \subseteq \R^d$ is a set of sphere packing centres with separation $r$ that is periodic with respect to some lattice $\Lambda$, then the density of the corresponding (periodic) sphere packing is given by
  \begin{equation}
    \frac{\lvert {X/\Lambda} \rvert}{\mathrm{Vol}(\R^d / \Lambda)} \cdot \mathrm{Vol}(B_d(0,r))
    \label{eq:periodic_packing_density_formula}
  \end{equation}
  where % the quotient in the numerator and denominator correspond to the orbits of the action by translation of $\Lambda$ on $X$ and $\R^d$ respectively, and $\Vol{\R^d / \Lambda}$ corresponds \emph{not} to the volume of the $(d + 1)$-dimensional torus obtained by taking the quotient in a topological sense but to that of the $d$-dimensional fundamental parallelepiped corresponding to any basis of $\Lambda$ (see Lemma~\ref{Basis.ofZlatticeBasis}).
  $\lvert X / \Lambda \rvert$ denotes the number of orbits of the additive action of $\Lambda$ on $X$ (cf. Lemma~\ref{PeriodicSpherePacking.instAddAction}) and $\Vol{\R^d / \Lambda}$ the covolume of $\Lambda$.
\end{lemma}
\begin{proof}
  Let $\Pa(X)$ denote the packing with centres at $X$. Definition~\ref{SpherePacking.density} tells us that
  \[
    \Delta_{\Pa(X)} = \limsup_{R \to \infty} \Delta_{\Pa(X)}(R) = \lim_{R \to \infty} \sup_{S \geq R} \frac{\Vol{\Pa(X) \cap B_d(0, S)}}{\Vol{B_d(0, S)}}
  \]
  The idea is to show that the sequence of suprema is constant by showing that the fraction is periodic. The way to do this would be to replace the $B_d(0, S)$ with a fundamental domain scaled by a factor proportional to $S$. How one would formalise this is not obvious at this juncture... [TODO: Flesh out mathematically as well]
\end{proof}

\begin{remark}
  Lemma~\ref{SpherePacking.density of periodic packing} can be thought of as the ``volume of spheres per fundamental domain": the number of spheres per fundamental domain is $\lvert {X/\Lambda} \rvert$, and the volume of each sphere is $\mathrm{Vol}(B_d(0,r))$.
\end{remark}

\begin{definition}\label{def-Periodic-sphere-packing-constant}\uses{SpherePacking.density, PeriodicSpherePacking}\notready
    The periodic sphere packing constant is defined to be
    $$ \Delta_{d}^{\text{periodic}} := \sup_{\substack{P \subset \R^d \\ \text{periodic packing}}} \Delta_P$$
\end{definition}

\begin{theorem}\label{periodic-packing-optimal}\uses{SpherePacking.density, def-Periodic-sphere-packing-constant}\notready
    For all $d$, the periodic sphere packing constant in $\R^d$ is equal to the sphere packing constant in $\R^d$.
\end{theorem}
\begin{proof}
  \todo{State this in Lean (ready).}
  \todo{Fill in proof here (see~\cite[Appendix A]{ElkiesCohn}}
\end{proof}

In other words, it suffices to compute and optimise the periodic sphere packing constant.

\subsection{Density of periodic packings}

In this subsection, we build up results about the density of periodic packings. In particular, the density of a periodic packing, defined as the limit of the periodic packing intersected with a growing ball centered at the origin, is equal to the density within any fundamental region of the period lattice.

% \todo{Actually add results here.}
