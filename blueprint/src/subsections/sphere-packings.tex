The Sphere Packing problem is a classic optimisation problem with widespread applications that go well beyond mathematics. The task is to determine the ``densest" possible arrangement of spheres in a given space. It remains unsolved in all but finitely many dimensions.

It was famously determined, in~\cite{Via2017}, that the optimal arrangement in $\mathbb{R}^8$ is given by the $E_8$ lattice. The result is strongly dependent on the Cohn-Elkies linear programming bound (Theorem 3.1 in~\cite{ElkiesCohn}), which, if a $\R^d \to \R$ function satisfying certain conditions exists, bounds the optimal density of sphere packings in $\R^d$ in terms of it. The proof in~\cite{Via2017} uses the theory of modular forms to construct a function that can be used to bound the density of all sphere packings in $\R^8$ above by the density of the $E_8$ lattice packing. This then allows us to conclude that no packing in $\R^8$ can be denser than the $E_8$ lattice packing.

\subsection{The Setup}

This subsection gives an overview for the setup of the problem, both informally and in Lean. Throughout this blueprint, $\R^d$ will denote the Euclidean vector space equipped with distance $\|\cdot\|$ and Lebesgue measure $\mathrm{Vol}(\cdot)$. For any $x\in\R^d$ and $r\in\R_{>0}$, we denote by $B_d(x,r)$ the open ball in $\R^d$ with center $x$ and radius $r$. While we will give a more formal definition of a sphere packing below (and in Lean), the underlying idea is that it is a union of balls of equal radius centred at points that are far enough from each other that the balls do not overlap.

Arguably the most important definition in this subsection is that of \emph{packing density}, which measures which portion of $d$-dimensional Euclidean space is covered by a given sphere packing. Taking the supremum over all packings gives what we refer to as the \emph{sphere packing constant}, which is the quantity we are interested in optimising.

\inputleannode{SpherePacking.balls}
\inputleannode{SpherePacking}

We now define a notion of density for bounded regions of space by considering the density inside balls of finite radius.

\inputleannode{SpherePacking.finiteDensity}

As intuitive as it seems to take the density of a packing to be the limit of the finite densities as the radius of the ball goes to infinity, it is not immediately clear that this limit exists. Therefore, we define the density of a sphere packing as a limit superior instead.

\inputleannode{SpherePacking.density}

We may now define the sphere packing constant, the quantity that the sphere packing problem requires us to compute.

\inputleannode{SpherePackingConstant}
