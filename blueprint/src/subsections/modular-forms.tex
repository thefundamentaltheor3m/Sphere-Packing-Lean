In this section, we recall and develop some theory of (quasi)modular forms.

\subsection{Modular forms and examples}

Let $\h$ be the upper half-plane $\{z\in\C\mid\Im(z)>0\}$.
\inputleannode{def:Gamma-1-Action}


Let $N$ be a positive integer.
\inputleannode{def:level-N-princ-cong-subgp}

\inputleannode{def:congruence-subgroup}

\inputleannode{def:Gamma-generators}

The following two lemmas tell us the group structure of $\Gamma(1) = \Gamma_1$ and $\Gamma(2)$, which we will use later on to define the theta forms.

\inputleannode{lemma:Gamma-1-generators}


\inputleannode{lemma:Gamma-2-generators}


Let $z\in\h$, $k\in\Z$, and $\left(\begin{smallmatrix}a&b\\c&d\end{smallmatrix}\right)\in\mathrm{SL}_2(\Z)$. We omit many of the proofs below when they exist in Mathlib already.
\inputleannode{def:automorphy-factor}

\inputleannode{lemma:automorphy-factor-chain-rule}


\inputleannode{def:slash-operator}

\inputleannode{lemma:slash-operator-chain-rule}


In particular, this lemma implies that if $\Gamma = \langle M_i \rangle_{i \in \mathcal{I}} \rangle$, then the slash action $F|\gamma$ is uniquely determined by the action of generators, i.e. $F|M_i$ and $F|M_i^{-1}$.

\inputleannode{lemma:slash-negI-even-weight}


\inputleannode{def:Mk}
By replacing condition $(3)$ above with (4) below defines the subspace of cusp forms, which we denote by $S_k(\Gamma)$.

\begin{enumerate}
	 \setcounter{enumi}{3}
	\item  For all $\gamma \in \mathrm{SL}_2(\mathbb{Z})$, and all  $0 < \epsilon$, there exists $A \in \mathbb{R}$ such that for all $z \in \mathbb{H}$, with $ A \le \mathrm{Im}(z)$, we have $|(f \mid_k \gamma) (z) |\le \epsilon$.
\end{enumerate}


Let us consider several examples of modular forms.
\inputleannode{def:Ek}
\inputleannode{lemma:Ek-is-modular-form}


Since modular forms are holomorphic and periodic they have a Fourier expansions. One thing that we
will need is the growth of the Fourier coefficients, which is given by the following lemma.

\inputleannode{lemma:mod_form_poly_growth}

For Eisenstein series, we can see this directly from the following:

\inputleannode{lemma:Ek-Fourier}



The infinite sum \eqref{eqn:Ek-definition} does not converge absolutely for $k=2$.
On the other hand, the expression \eqref{eqn:Ek-Fourier} converges to a holomorphic function on the upper half-plane and we will take it as a definition of $E_2$ (See Definition \ref{def:E2} below).


The discriminant form is a unique normalized cusp form of weight 12, which can be defined as:
\inputleannode{def:disc-definition}

This product formula allows us to prove positivity of $\Delta(it)$ for $t > 0$ later. But we need to first check its a modular form. For this we first need some definitions/ results.

We define it as a $q$-series, which gives a holomorphic function on $\mathfrak{H}$.
\inputleannode{def:E2}

\inputleannode{lemma:E2-transform-S}



More generally, we have
\inputleannode{lemma:E2-transform-general}


\inputleannode{def:dedekind_eta}

\inputleannode{lemma:dedekind_eta_transformation}



\inputleannode{lemma:disc-cuspform}




\inputleannode{lemma:disc-E4E6}


%Note that the RHS of \eqref{eqn:disc-logder} is equal to the $E_2(z)$.
%As a side note, we can also consider defining $\Delta$ as \eqref{eqn:disc-prodformula}, and prove that it coincides with \eqref{eqn:disc-definition}.
%Such an argument can be found in \cite[Section 2.4]{Bruinier}.

\inputleannode{cor:disc-pos}


The following nonvanishing result, which directly follows from \Cref{def:disc-definition}, will be used in the construction of the magic function.
\inputleannode{cor:disc-nonvanishing}



A key fact in the theory of modular forms is that the spaces $M_k(\Gamma)$ are finite-dimensional.
To prove this we will do use the following non-standard proof. First we have the following result.

\inputleannode{thm:nonpos_wt}



\inputleannode{thm:lvl1_dims}


\inputleannode{thm:dim-mf-general-level}


% the following theorem:
% \begin{theorem}\label{theorem-Mk-finite-dimensional}\uses{def:Mk}
%     The spaces $M_k(\Gamma)$ are finite dimensional.
% \end{theorem}
% \begin{proof}
%   In this project, we only require the theorem for $\Gamma = \Gamma_1$ and $\Gamma = \Gamma_2$. For the proof, see~\cite{Serre73}.
% \end{proof}
% A key fact in the theory of modular forms is that they form a finite-dimensional vector space (for a fixed weight and level).
% Especially, we use the following dimension results.

\inputleannode{cor:dim-mf}


Another examples of modular forms we would like to consider are \emph{theta functions} \cite[Section~3.1]{1-2-3}.
\inputleannode{def:th00-th01-th10}

For convenience, we use the following notations for the fourth powers of the theta functions.
\inputleannode{def:H2-H3-H4}
Note that we only need these fourth powers to define \eqref{def:b-definition}.

The group $\Gamma_1$ is generated by the elements $T=\left(\begin{smallmatrix}1&1\\0&1\end{smallmatrix}\right)$, $S=\left(\begin{smallmatrix}0&1\\-1&0\end{smallmatrix}\right)$, and $-I = \left(\begin{smallmatrix}-1&0\\0&-1\end{smallmatrix}\right)$ (\Cref{lemma:Gamma-1-generators}), and the transformation of functions under $\Gamma(2)$ is determined by that under $\Gamma_1$ (by \Cref{lemma:slash-operator-chain-rule}).
The following lemma shows how the theta functions (and their powers) transform under the slash action of these matrices.

\inputleannode{lemma:theta-transform-S-T}


Using the above identities, we can prove that these are modular forms.
\inputleannode{lemma:theta-slash-invariant}


\inputleannode{lemma:theta-bounded-im-infty}


\inputleannode{lemma:theta-modular}



They have Fourier expansions as follows.
\begin{proposition}\label{prop:H2-fourier}\uses{def:H2-H3-H4}
$H_2$ admits a Fourier series of the form
\begin{equation}
    H_2(z) = \sum_{n \ge 1} c_{H_2}(n) e^{\pi i n z}
\end{equation}
for some $c_{H_2}(n) \in \R_{\ge 0}$, with $c_{H_2}(1) = 16$ and $c_{H_2}(n) = O(n^k)$ for some $k \in \N$.
\end{proposition}
\begin{proof}
We have
\begin{align}
    H_2(z) &= \Theta_2(z)^4 \\
    &= \left(\sum_{n \in \Z} e^{\pi i (n + \frac{1}{2})^{2} z}\right)^{4} \\
    &= \left(2\sum_{n \ge 0} e^{\pi i (n + \frac{1}{2})^{2} z}\right)^{4} \\
    &= \left(2 e^{\pi i z / 4} + 2 \sum_{n \ge 1} e^{\pi i (n^2 + n + \frac{1}{4}) z}\right)^{4} \\
    &= 16 e^{\pi i z}\left(1 + \sum_{n \ge 1} e^{\pi i (n^2 + n)z}\right)^{4} \\
    &= 16 e^{\pi i z} + \sum_{n \ge 2} c_{H_2}(n) e^{\pi i n z} \\
    &= \sum_{n \ge 1} c_{H_2}(n) e^{\pi i n z}.
\end{align}
\end{proof}

\begin{proposition}\label{prop:H3-fourier}\uses{def:H2-H3-H4}
$H_3$ admits a Fourier series of the form
\begin{equation}
    H_3(z) = \sum_{n \ge 0} c_{H_3}(n) e^{\pi i n z}
\end{equation}
for some $c_{H_3}(n) \in \R_{\ge 0}$ with $c_{H_3}(0) = 1$ and $c_{H_3}(n) = O(n^k)$ for some $k \in \N$.
Especially, $H_3$ is not cuspidal.
\end{proposition}
\begin{proof}
We have
\begin{equation}
    H_3(z) = \Theta_3(z)^{4} = \left(\sum_{n \in \Z} e^{\pi i n^2 z}\right)^{4} = \left(1 + 2 \sum_{n \ge 1} e^{\pi i n^2 z}\right)^{4} = 1 + O(e^{\pi i z}).
\end{equation}
\end{proof}

\begin{proposition}\label{prop:H4-fourier}\uses{def:H2-H3-H4}
$H_4$ admits a Fourier series of the form
\begin{equation}
    H_4(z) = \sum_{n \ge 0} c_{H_4}(n) e^{\pi i n z}
\end{equation}
for some $c_{H_4}(n) \in \R$ with $c_{H_4}(0) = 1$ and $c_{H_4}(n) = O(n^k)$ for some $k \in \N$.
Especially, $H_4$ is not cuspidal.
\end{proposition}


We also have a nontrivial relation between these theta functions.
\inputleannode{lemma:jacobi-identity}


These are also related to $E_4$, $E_6$, and $\Delta$ as follows.
\begin{lemma}\label{lemma:lv1-lv2-identities}\uses{lemma:theta-transform-S-T, lemma:theta-modular, lemma:disc-cuspform}
We have
\begin{align}
    E_4 &= \frac{1}{2}(H_{2}^{2} + H_{3}^{2} + H_{4}^{2}) = H_{2}^{2} + H_{2}H_{4} + H_{4}^{2} \label{eqn:e4theta} \\
    E_6 &= \frac{1}{2} (H_{2} + H_{3})(H_{3} + H_{4}) (H_{4} - H_{2}) = \frac{1}{2}(H_2 + 2H_4)(2H_2 + H_4)(H_4 - H_2) \label{eqn:e6theta} \\
    \Delta &= \frac{1}{256} (H_{2}H_{3}H_{4})^2. \label{eqn:disctheta}
\end{align}
\end{lemma}
\begin{proof}
We can prove these similarly as Lemma \ref{lemma:jacobi-identity}.
Right hand sides of \eqref{eqn:e4theta}, \eqref{eqn:e6theta}, and \eqref{eqn:disctheta} are all modular forms of level $\Gamma_1$ and desired weights, where \eqref{eqn:disctheta} is a cusp form since $H_2$ is.
Now the identities follow from the dimension calculations $\dim M_4(\Gamma_1) = \dim M_6(\Gamma_1) = \dim S_{12}(\Gamma_1) = 1$ and comparing the first nonzero $q$-coefficients.
\end{proof}

The \emph{strict} positivity of Jacobi theta functions might needed later.
\inputleannode{cor:theta-pos}


\subsection{Quasimodular forms and derivatives}

Morally, quasimodular forms can be thought as \emph{modular forms with differentiations}.
It can be defined formally as follows:
Let $f: \mathfrak{H} \to \C$ be a holomorphic function, and let $k$ and $s \ge 0$ be integers.
The function $f$ is a \emph{quasimodular form of weight $k$, level $\Gamma$, and depth $s$} if there exist holomorphic functions $f_0, \dots, f_s : \mathfrak{H} \to \C$ such that
\begin{equation}\label{eqn:quasimod-def}
    (f|_{k}\gamma)(z) = (cz + d)^{-k} f\left(\frac{az + b}{cz + d}\right) = \sum_{j=0}^{s} f_j(z) \left(\frac{c}{cz + d}\right)^j
\end{equation}
for all $z \in \mathfrak{H}$ and $\gamma = \left(\begin{smallmatrix} a&b\\c&d \end{smallmatrix}\right) \in \Gamma$.

By taking $\gamma = \left(\begin{smallmatrix} 1 & 0 \\ 0 & 1 \end{smallmatrix}\right)$, one can check that we should have $f_0 = f$. Thus, a quasimodular form of depth $0$ is just a modular form of same weight and level.
Also, it is easy to see that the space of quasimodular forms is closed under the normalized derivative.

In this project, we are \emph{not} going to formalize the above definition of quasimodular forms.
Instead, we only use $E_2$ (defined in \Cref{def:E2}) and define the normalized derivative operator $D$ (\Cref{def:derivative}) and the Serre derivative $\partial_k$ (\Cref{def:serre-der}) as the main tools to work with quasimodular forms.

\inputleannode{def:derivative}

Normalizing the derivative by $1/(2 \pi i)$ as in \eqref{eqn:derivative} is standard in the theory of modular forms, since it makes the $q$-expansion coefficients nicer.
\inputleannode{lemma:der-q-series}


The most important example of quasimodular form is the weight 2 Eisenstein series $E_2$ (\Cref{def:E2}).
Using it, we can define the \emph{Serre derivative} of a quasimodular form.

\inputleannode{def:serre-der}

\inputleannode{thm:serre-der-equiv-action}


As a direct consequence of Theorem \ref{thm:serre-der-equiv-action}, we can check that the Serre derivative preserves the modularity of a modular form.
\inputleannode{thm:serre-der-modularity}


\begin{remark}
More generally, the following theorem holds: if $F$ is a quasimodular form of weight $k$ and depth $s$, then $\partial_{k-s}F$ is a quasimodular form of weight $k + 2$ \emph{and depth $\le s$} of the same level. We will not prove this here.
\end{remark}

\inputleannode{thm:ramanujan-formula}

\begin{corollary}\label{cor:logder-disc-E2}\uses{thm:ramanujan-formula, def:disc-definition}
\begin{equation}\label{eqn:logder-disc-E2}
    \Delta' = E_2 \Delta.
\end{equation}
\end{corollary}
\begin{proof}
By Ramanujan's formula \eqref{eqn:DE4} and \eqref{eqn:DE6},
\begin{equation}
\Delta' = \frac{3 E_4^2 E_4' - 2 E_6 E_6'}{1728} = \frac{1}{1728} \left(3 E_4^2 \cdot \frac{E_2 E_4 - E_6}{3} - 2 E_6 \cdot \frac{E_2 E_6 - E_4^2}{2}\right) = \frac{E_2(E_4^3 - E_6^2)}{1728} = E_2\Delta.
\end{equation}
\end{proof}

Similar argument allow us to compute (Serre) derivatives of $H_2, H_3, H_4$.
\begin{proposition}\label{prop:theta-der}\uses{def:serre-der, lemma:theta-transform-S-T, lemma:jacobi-identity}
We have
\begin{align}
    H_2' &= \frac{1}{6} (H_{2}^{2} + 2 H_{2} H_{4} + E_2 H_2) \label{eqn:H2-der}\\
    H_3' &= \frac{1}{6} (H_{2}^{2} - H_{4}^{2} + E_2 H_3) \label{eqn:H3-der}\\
    H_4' &= -\frac{1}{6} (2H_{2} H_{4} + H_{4}^{2} - E_2 H_4) \label{eqn:H4-der}
\end{align}
or equivalently,
\begin{align}
    \partial_{2} H_{2} &= \frac{1}{6} (H_{2}^{2} + 2 H_{2} H_{4}) \label{eqn:H2-serre-der} \\
    \partial_{2} H_{3} &= \frac{1}{6} (H_{2}^{2} - H_{4}^{2}) \label{eqn:H3-serre-der} \\
    \partial_{2} H_{4} &= -\frac{1}{6} (2H_{2} H_{4} + H_{4}^{2}) \label{eqn:H4-serre-der}
\end{align}
\end{proposition}
\begin{proof}
Equivalences are obvious from the definition of the Serre derivative.
Define $f_{2}, f_{3}, f_{4}$ be the differences of the left and right hand sides of \eqref{eqn:H2-serre-der}, \eqref{eqn:H3-serre-der}, \eqref{eqn:H4-serre-der}.
\begin{align}
    f_{2} &:= \partial_{2} H_{2} - \frac{1}{6} H_{2}(H_{2} + 2H_{4}) \\
    f_{3} &:= \partial_{2} H_{3} - \frac{1}{6} (H_{2}^2 - H_{4}^2) \\
    f_{4} &:= \partial_{2} H_{4} + \frac{1}{6} H_{4}(2H_{2} + H_{4}).
\end{align}
Then these are a priori modular forms of weight $4$ and level $\Gamma(2)$, and our goal is to prove that they are actually zeros.
By Jacobi's identity \eqref{eqn:jacobi-identity}, we have $f_{2} + f_{4} = f_{3}$.
Also, the transformation rules of $H_2, H_3, H_4$ give
\begin{align}
    f_{2}|_{S} &= -f_{4} \\
    f_{2}|_{T} &= -f_{2} \\
    f_{4}|_{S} &= -f_{2} \\
    f_{4}|_{T} &= f_{3} = f_{2} + f_{4}.
\end{align}
Now, define
\begin{align}
    g &:= (2 H_2 + H_4) f_2 + (H_2 + 2 H_4) f_4 \\
    h &:= f_{2}^{2} + f_{2}f_{4} + f_{4}^{2}.
\end{align}
Then one can check that both $g$ and $h$ are invariant under the actions of $S$ and $T$, hence they are modular forms of level $1$.
Also, by analyzing the limit of $g$ and $h$ as $z \to i \infty$, one can see that $g$ and $h$ are cusp forms, hence $g = h = 0$ by \eqref{eqn:dimS6} and \eqref{eqn:dimS8}.
This implies
\begin{align}
    3 E_4 f_2^{2} &= 3 (H_2^2 + H_2 H_4 + H_4^2) f_2^{2} = ((2 H_2 + H_4)^{2} - (2H_2 + H_4)(H_2 + 2H_4) + (H_2 + 2H_4)^{2}) f_2^{2}\\
    &= (2 H_2 + H_4)^{2} (f_2^2 + f_2 f_4 + f_4^2) = 0
\end{align}
and by considering $q$-series ($E_4$ has an invertible $q$-series), we get $f_2 = 0$.
\end{proof}

\inputleannode{thm:serre-der-prod-rule}


We also have the following useful theorem for proving positivity of quasimodular forms on the imaginary axis, which is \cite[Proposition 3.5, Corollary 3.6]{Lee}.
\begin{theorem}\label{thm:anti-serre-der-pos}\uses{def:serre-der, cor:logder-disc-E2}
Let $F$ be a holomorphic quasimodular cusp form with real Fourier coefficients.
Assume that there exists $k$ such that $(\partial_{k}F)(it) > 0$ for all $t > 0$.
If the first Fourier coefficient of $F$ is positive, then $F(it) > 0$ for all $t > 0$.
\end{theorem}
\begin{proof}
By \eqref{eqn:logder-disc-E2}, we have
\begin{align}
    \frac{\dd}{\dd t} \left( \frac{F(it)}{\Delta(it)^{\frac{k}{12}}}\right)
    &= (-2 \pi) \frac{F'(it) \Delta(it)^{\frac{k}{12}} - F(it) \frac{k}{12} E_{2}(it) \Delta(it)^{\frac{k}{12}}}{\Delta(it)^{\frac{k}{6}}} \\
    &= (-2 \pi) \frac{(\partial_{k} F)(it)}{\Delta(it)^{\frac{k}{12}}}  < 0,
\end{align}
hence
\[
t \mapsto \frac{F(it)}{\Delta(it)^{\frac{k}{12}}}
\]
is monotone decreasing.
Because of the assumption on the positivity of the first nonzero Fourier coefficient of $F$, $F(it) > 0$ for sufficiently large $t$ since
\[
F = \sum_{n \geq n_{0}} a_{n} q^{n} \Rightarrow e^{2 \pi n_{0} t} F(it) = a_{n_{0}} + e^{-2 \pi t}\sum_{n\geq n_{0} + 1} a_{n} e^{-2 \pi (n - n_{0} - 1)t}
\]
and $\lim_{t \to \infty} e^{2 \pi n_{0}t} F(it) = a_{n_0} > 0$, hence the result follows.
\end{proof}
