\subsection{Bounds on Finite Density of Periodic Prcking}

In this subsection, we build up results about the density of periodic packings. In particular, the density of a periodic packing, defined as the limit of the periodic packing intersected with a growing ball centered at the origin, is equal to the density within any fundamental region of the period lattice. The strategy is to prove lower and upper bounds for the number of lattice points in a ball in terms of the volume of the ball, correct up to the highest order term. Taking limit gives the correct density!

Below, let $X \subseteq \R^d$ be a set of periodic packing centers with respect to the lattice $\Lambda \subset \R^d$. We write $kX \coloneqq \{kv : v \in X\}$.

\begin{definition}\label{fd}\lean{}\uses{}
  Let $\Lambda \subset \R^d$ be a lattice. A set $\mathcal{D} \subseteq \R^d$ is a \emph{fundamental domain} of $\Lambda$ such that for all distinct $x, y \in \Lambda$, we have $(x + \mathcal{D}) \cap (y + \mathcal{D}) = \emptyset$ (disjointness) and $\bigcup_{x \in \Lambda} x + \mathcal{D} = \R^d$ (tiling).
\end{definition}

\begin{lemma}\label{exists-bounded-fd}\lean{}\uses{fd}
  There always exists a \emph{bounded} fundamental region $\mathcal{D}$ of $\Lambda$.
\end{lemma}
\begin{proof}
  Since lattices have $\Z$-bases, there exists a set of vectors $\mathcal{B} \subseteq \R^d$ such that $\Lambda = \mathrm{span}_{\Z}(\mathcal{B})$. We claim that $\mathcal{D}_{\Lambda} = \{\sum_i c_i\mathcal{B}_i \subseteq \R^n : c_i \in [0, 1)^n\}$ is a fundamental domain. The rest exists in Mathlib already so I don't bother elaborating here :) From the definition, we see that for $v = \sum_i c_i\mathcal{B}_i \in \mathcal{D}_{\Lambda}$, we have $\|v\| \leq \sum_i \|c_i\mathcal{B}_i\| \leq \sum_i \|\mathcal{B}_i\|$, which is a constant. Hence, $\mathcal{D}_{\Lambda}$ is bounded.
\end{proof}

We denote by $L$ the bound of norm of vectors in the fundamental domain $\mathcal{D}$.

\begin{lemma}\label{exists-fd-coset-contains}\lean{}\uses{fd}
  For all vectors $v \in \R^d$ there exists a unique lattice point $x \in \Lambda$ such that $v \in x + \mathcal{D}$.
\end{lemma}
\begin{proof}
  By the tiling property of the fundamental domain, we have $v \in \bigcup_{x \in \Lambda} (x + \mathcal{D})$. By definition, this means there exists a lattice point $x \in \Lambda$ such that $v \in x + \mathcal{D}$. To show that it is unique, suppose that $v \in (x + \mathcal{D}) \cap (y + \mathcal{D})$ for distinct $x \neq y \in \Lambda$. By the disjointness property, $v \in \emptyset$, contradiction.
\end{proof}

\begin{proof}[Proof of \cref{lattice-points-bounds}]\label{lattice-points-bounds-proof}\uses{fd, exists-fd-coset-contains}
  For the first inequality, it suffices to prove that $\mathcal{B}_d(R - L) \subseteq \bigcup_{x \in \Lambda \cap \mathcal{B}_d(R)} (x + \mathcal{D})$, since the cosets on the right are almost disjoint. For a vector $v \in \mathcal{B}_d(R - L)$, we have $\|v\| < R - L$ by definition. By Lemma \cref{exists-fd-coset-contains}, there exists a lattice point $x \in \Lambda$ such that $v \in x + \mathcal{D}$. Rearranging gives $v - x \in \mathcal{D}$, which means $\|v - x\| \leq L$. By the triangle inequality, $\|x\| < R$, i.e. $x \in \mathcal{B}_d(L)$, concluding the proof.

  For the second inequality, we prove that $\bigcup_{x \in \Lambda \cap \mathcal{B}_d(R)} (x + \mathcal{D}) \subseteq \mathcal{B}_d(R + L)$. Consider a vector $x \in \Lambda \cap \mathcal{B}_d(R)$ and a vector $y \in x + \mathcal{D}$. From above, we know $\|x\| < R$ and $\|y - x\| \leq L$, so $\|y\| < R + L$, concluding the proof.
\end{proof}
