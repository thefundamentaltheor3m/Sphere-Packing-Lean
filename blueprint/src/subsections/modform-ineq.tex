\section{Proof of Theorem \ref{thm:g}}\label{sec: g}
Our proof of the Theorem~\ref{thm:g} relies on the following two inequalities for modular objects.
\begin{proposition}\label{prop:ineqA}\uses{lemma:ineqABnew-equiv, lemma:F-G-phi-psi-identities, lemma:F-G-pos, cor:ineqAnew}
Consider the function $A:(0,\infty)\to\C$ defined as
\begin{equation}\label{eqn:defA}
A(t):=-t^2\phi_0(i/t)-\frac{36}{\pi^2}\,\psi_I(it).
\end{equation}
Then
\begin{equation}\label{eqn:ineqA}
    A(t) < 0
\end{equation}
for all $t > 0$.
\end{proposition}

\begin{proposition}\label{prop:ineqB}\uses{lemma:ineqABnew-equiv, lemma:F-G-phi-psi-identities, cor:ineqBnew}
Consider the function $B:(0,\infty)\to\C$ defined as
\begin{equation}\label{eqn:defB}
    B(t) := -t^2\phi_0(i/t)+\frac{36}{\pi^2}\,\psi_I(it)
\end{equation}
Then
\begin{equation}\label{eqn:ineqB}
    B(t) > 0
\end{equation}
for all $t > 0$.
\end{proposition}

Here we formalize the proof of the inequalities by Lee \cite{Lee}.
First, we can rewrite the inequality in \ref{prop:ineqA} as follows.

\inputleannode{def:FG-definition}

\begin{lemma}\label{lemma:F-G-phi-psi-identities}\uses{def:FG-definition, lemma:psi-new}
We have
\begin{align}
    \phi_0 &= \frac{F}{\Delta} \label{eqn:phi0-F} \\
    \psi_S &= -\frac{1}{2} \frac{G}{\Delta}\label{eqn:psiS-G}
\end{align}
\end{lemma}
\begin{proof}
\eqref{eqn:phi0-F} is clear.
\eqref{eqn:psiS-G} is already shown in Lemma \ref{lemma:psi-new}.
\end{proof}

\begin{lemma}\label{lemma:ineqABnew-equiv}\uses{lemma:F-G-phi-psi-identities, def:psiI-psiT-psiS, cor:disc-pos}
Inequality \eqref{eqn:ineqA} and \eqref{eqn:ineqB} are equivalent to
\begin{align}
    F(it) + \frac{18}{\pi^2} G(it) > 0 \label{eqn:ineqAnew} \\
    F(it) - \frac{18}{\pi^2} G(it) > 0 \label{eqn:ineqBnew}
\end{align}
respectively.
\end{lemma}
\begin{proof}
By \eqref{eqn:psiS-define},
\begin{equation}
    \psi_I(it) = (\psi_S|_{-2}S)(it) = (it)^{2}\psi_S\left(-\frac{1}{it}\right) = -t^2 \psi_S\left(\frac{i}{t}\right).
\end{equation}
Combined with Lemma \ref{lemma:F-G-phi-psi-identities} we can rewrite \eqref{eqn:ineqA} as
\begin{equation}
    A(t) = -t^2 \phi_0\left(\frac{i}{t}\right) + \frac{36}{\pi^2} \psi_S\left(\frac{i}{t}\right) < 0 \Leftrightarrow \frac{F(it)}{\Delta(it)} + \frac{18}{\pi^2} \frac{G(it)}{\Delta(it)} > 0
\end{equation}
for $t > 0$, which is equivalent to \eqref{eqn:ineqAnew} by Corollary \ref{cor:disc-pos}.
Equivalences of \eqref{eqn:ineqB} and \eqref{eqn:ineqBnew} follows similarly; just change the sign.
\end{proof}


Now, the first inequality \eqref{eqn:ineqAnew} follows from the positivity of each $F(it)$ and $G(it)$.

\inputleannode{lemma:F-G-pos}


\inputleannode{cor:ineqAnew}


To prove the second inequality \eqref{eqn:ineqBnew}, we need some identities satisfied by $F$ and $G$.
\inputleannode{lemma:FG-de}


\inputleannode{cor:MLDE-pos}


The second inequality \eqref{eqn:ineqBnew} follows from the following two observations.
Since $G(it) > 0$ for all $t > 0$, we can define the quotient
\begin{equation}\label{eqn:Q}
    Q(t) := \frac{F(it)}{G(it)}
\end{equation}
as a function on $(0, \infty)$.

\inputleannode{lemma:Qlim}


\begin{lemma}\label{lemma:log-der-inf}\uses{lemma:der-q-series}
Let $F$ be a quasimodular form where the vanishing order of $F$ at infinity is $n_0 > 0$, i.e. $F(z) = \sum_{n \geq n_0} a_n q^{n}$ with $a_{n_0} \ne 0$. Then
\begin{equation}
    \lim_{t \to \infty} \frac{F'(it)}{F(it)} = n_0.
\end{equation}
\end{lemma}
\begin{proof}
    By Lemma \ref{lemma:der-q-series},
    \begin{equation}
        \lim_{t \to \infty} \frac{F'(it)}{F(it)} = \lim_{t \to \infty} \frac{\sum_{n \ge n_0} n a_n e^{-2 \pi n t}}{\sum_{n \ge n_0} a_n e^{-2 \pi n t}} = \lim_{t \to \infty} \frac{n_0 a_{n_0} e^{-2 \pi n_0 t} + O(e^{-2 \pi (n_0 + 1) t})}{a_{n_0} e^{-2 \pi n_0 t} + O(e^{-2 \pi (n_0 + 1) t})} = n_0.
    \end{equation}
\end{proof}

\inputleannode{prop:Qdec}



\inputleannode{cor:ineqBnew}



Finally, we are ready to prove Theorem~\ref{thm:g}.
\begin{theorem}\label{thm:g1}
\uses{prop:a-fourier, prop:b-fourier, prop:a-double-zeros, prop:b-double-zeros, prop:ineqA, prop:ineqB, prop:a0, prop:b0}
The function
$$g(x):=\frac{\pi\,i}{8640}a(x)+\frac{i}{240\pi}\,b(x)$$
satisfies conditions \eqref{eqn:g1}--\eqref{eqn:g3}.
\end{theorem}
\begin{proof}
First, we prove that \eqref{eqn:g1} holds. By Propositions~\ref{prop:a-double-zeros} and \ref{prop:b-double-zeros} we know that for $r>\sqrt{2}$
\begin{equation}\label{eqn:g A} g(r)=\frac{\pi}{2160}\,\sin(\pi r^2/2)^2\,\int\limits_0^\infty A(t)\,e^{-\pi r^2 t}\,dt\end{equation}
where $$A(t)=-t^2\phi_0(i/t)-\frac{36}{\pi^2}\,\psi_I(it).$$
from the Proposition~\ref{prop:ineqA} we know that $A(t)<0\quad\mbox{for}\;t\in(0,\infty).$
Therefore identity \eqref{eqn:g A} implies \eqref{eqn:g1}.

Next, we prove \eqref{eqn:g2}. By Propositions~\ref{prop:a-another-integral} and~\ref{prop:b-another-integral} we know that for $r>0$
\begin{equation}\label{eqn:g B} \widehat{g}(r)=\frac{\pi}{2160}\,\sin(\pi r^2/2)^2\,\int\limits_0^\infty B(t)\,e^{-\pi r^2 t}\,dt\end{equation}
where $$B(t)=-t^2\phi_0(i/t)+\frac{36}{\pi^2}\,\psi_I(it).$$


Finally, the property \eqref{eqn:g3} readily follows from Proposition~\ref{prop:a0} and Proposition~\ref{prop:b0}.
This finishes the proof of Theorems~\ref{thm:g1} and~\ref{thm:g}.
\end{proof}
